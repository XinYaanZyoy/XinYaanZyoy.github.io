\documentclass[a4paper]{article}

\usepackage[pdfusetitle]{hyperref}
\hypersetup{pdfauthor={XinYaanZyoy}, 
            pdftitle={Essay on BG:3.35},
            pdfsubject={Hinduism}, 
            pdfkeywords={Gita, India, Independence, karmayoga},
            pdfproducer={xeLatex on Ubuntu with hyperref},
            pdfcreator={xeLatex}
            }
\usepackage[nottoc,numbib]{tocbibind}

% Multilang
% ________
\usepackage{polyglossia}
\setmainlanguage{english}
\setotherlanguages{hindi,sanskrit}
\newfontfamily\devanagarifont[Script=Devanagari]{Samyak Devanagari}
\newfontfamily{\Guja}[Script=Gujarati]{Noto Sans Gujarati}
% ________

\title{Essays on Karma-yoga and Indian Independence Movement \\ 
\large Their Philosophical and Historical Connections \\ particularly related to BG:3.35}

\author{XinYaanZyoy\footnote{\href{https://xinyaanzyoy.github.io/}{https://xinyaanzyoy.github.io/}} \\ \href{mailto:XinYaanZyoy@gmail.com}{XinYaanZyoy@gmail.com}}

\date{\today\footnote{it's my birth date!}}

\begin{document}

\maketitle

\tableofcontents

\begin{abstract}
These are short (about 700 words each) essays on Karma-Yoga, Indian Independence Movement, 
and their philosophical and historical connections. there also is a try to converge towards 
philosophical and historical connections of Indian Independence to a specific sloka of Srimad Bhagavad Gita, 
namely verse 35 of chapter 3, written from a perspective of the "Indian Identity", that is an identity which is 
supposed to be of "the people" but is instead, somehow, of "the culture", "the land", and "the global relations". 
The identity of the aforementioned kind played a huge role in the "Indian independence movement". And as with any revolutionary 
movements in the history of mankind, there were justifications of "Good" and "Evil" based on "Religions" and "Philosophies", 
in the case of Revolutionary Events for Indian Independence particularly "Vedic" Religions and Philosophies played a role as 
a strong foundation of these justifications which even today are praised and criticized all over the globe.
\end{abstract}
    
\newpage
\section{Introduction}
I am planning to write a structured commentary on Srimad Bhagavad Gita when I become sufficiently apt for making philosophical 
arguments that are more structured, coherent, and "useful". I'm just a kid, as of now, to be able to do such a thing since my 
current arguments mostly rely on the string of thoughts along the direction in which the conflict of "emotional and rational" 
existence takes me to, which most of the time are not "useful", though is of immense quality, rigor, and structure. so this 
planning still has to wait for quite a long time!

The choice of the Sloka and the theme thereof is rather an accident! These essays are meant to assist someone\footnote{A special 
one!} I know in doing their assigned activity at one of the youth centers of Swadhyay Parivar\footnote{
\href{https://en.wikipedia.org/wiki/Swadhyaya\_Movement}{https://en.wikipedia.org/wiki/Swadhyaya\_Movement}}, 
which is the reason why there's, a section on composition of all the essays of next section in Gujarati.

I find many paradoxes and misuses of the so-called "multiple interpretation" literatures of the Vedic period of Bharat, 
but instead, I try to focus on the positive and effective aspects of those events, particularly related to BG:3.35 which 
can be taught to school kids as an Inspiration of the Indian history, culture, and Identity, which by no means what I wish to happen, 
because I believe in structured schooling, but taking into account of the practical and real-life schooling, this is exactly 
I wish to happen, and at some extent is already happening, in fact, I myself am a product of one such initiative at my high-school.

Due to lack of a medium of communication that is structured and rigorous, and since the traditional medium, namely that of 
speech, and writing are more linear than they should be to accommodate the communication of the type of the deeper 
subjects, like of Srimad Bhagavad Gita, I tend to be more "verbosely informative" at the same time "concise", so that I don't end 
up writing gibberish and creating a multi-thread of thoughts in the mind of the reader.

To write these Essays from multiple interpretations, I think I need a more in-depth understanding of every interpretation, 
which I don't think I have as of now, so I read, actually skimmed through, the references\footnote{well, I don't claim to have 
them read the way I usually do, because this time I had to write these essays within a prescribed time so!}, and several 
webpages and encyclopedias, it's to be noted therefore that, I do not claim to have understood Bhagavad Gita nor do I claim to have read 
it in its full generality, so my words might be ambiguous, paradoxical, and even maligned from the "real" words, I ask the reader 
to forgive me for this.

\newpage

\section{Essays}
% ब्रह्मात्सर्वमुदच्यते। read: brahm-aat-sarvam-udach-ayate meaning: out of the 
Having paid reverence to Brahman\footnote{\href{https://en.wikipedia.org/wiki/Brahman}{https://en.wikipedia.org/wiki/Brahman}} 
(\begin{sanskrit}ब्रह्मन्\end{sanskrit}), the eternal and metaphysical ocean which is the \begin{sanskrit}साध्य\end{sanskrit}(the goal), 
the \begin{sanskrit}साधन\end{sanskrit}(the way to reach the goal) and the \begin{sanskrit}साध्यार्थ\end{sanskrit}(the purpose of the goal) of 
this existence, I, XinYaanZyoy, set forth these Essays on The Bhagvad-Gita, The Karama-Yoga, The Indian Independence Movement, 
The Freedom Fighters, and the history, philosophies, and connections thereof. I sincerely beg for help and an apology from that 
manifestation of Brahman, who is the goddess of sciences, namely Brahmi or Saraswati, to make these essays "useful", and for if I 
may not be able to do it respectively.

\subsection{On The Indian Independance Movement}
Spanning from the mid of 19th to the mid of 20th century The Indian Independance Movement (hereafter abbreviated as IIM), which according to Wikipedia\footnote{\href{https://en.wikipedia.org/wiki/Indian\_independence\_movement}
{https://en.wikipedia.org/wiki/Indian\_independence\_movement}}, was a series of historic events with the ultimate aim of ending 
British rule in India. most of the 19th century events revolved around leaders "seeking the fundamental rights" for their people, 
while a part of the beginning of 20th century mostly revolved around "political self-rule" which was led by many leaders including Bal Gangadhar Tilak, 
the first leader since the genisis of IIM who made "Swaraj" as the destiny of the nation, and the other part of the beginning of 20th century mostly revolved around 
Mahatma Gandhi and his principles, especially that of non-violence which he had learned from Bhagavad Gita.

As mentioned earlier, The Bhagavad Gita played a role in The IIM, many  freedom activists, freedom fighters, nationalists, 
spiritual leaders, and even common men took refuge in understanding what's 
right and should be done, and in establishing what's wrong and shouldn't be done, and took action accordingly to play their part 
in the IIM. As is the case with any Ancient Indian literature, Srimad Bhagavad Gita can be interpreted in 
multiple ways, and so did all these people who played their part, we find interpretations in which violence to establish freedom 
is considered just as important as non-violence in other's interpretations, it is, therefore, an amusing 
journey in the past of the "Republic of India".

Most events of IIM, generally speaking, were mass-based that encompassed various sections of society. 
We find various sections\footnote{I'm writing just a few of those here, not that these were the only existing sections, and also I write only 
one name per section as a representative, not that there were no other, also I might have not included many important figures of 
whose books I might have referenced.} of society, be it individually or in group, playing their part in the IIM. we saw a section of 
youth like Bhagat Singh, a section of females like Sarojini Naidu, a section of poets and writers like Rabindranath Tagore and Bankim Chandra Chattopadhyay,
and even a section of Philosophers like Aurobindo Ghosh. if we dig deeper into what caused them to do such
seemingly impossible acts, we find that they all had some sort of philosophies they were basing all of their acts on. the philosophies 
of this sort came from religious texts, though not to say that there were no other kind of leaders who did all their actions
on the base of humanity. But most of the leadership, or at least the one I'm interested in writing about in these essays,
emerged from philosophies that arise from religious texts, especially the Bhagavad Gita, this helped them understand what's right and what's not. most of these philosophies were subjective, in a sense that
they were interpreting the same literature from their own perspectives, which caused an amusing story to be told, namely that of how
a single book made two people interpret it extremely differently that one goes on the path of violence while the other on that of non-violence.

It's a bit difficult to reach a conclusive philosophy on the events of the IIM, especially within the scope of these essays and 
length thereof, I, therefore, have divided (a person can be in more than two classes though!) people who contributed in the IIM 
into two major classes, namely that of Nationalist Freedom Fighters and Nationalist Spiritual Leaders. though it's to be noted 
that by no means do I claim that each person in a class had the "same" interpretation, but instead I claim that we can abstract 
away most of the details and find, within the scope of Karma Yoga, that they did have the "same" interpretation. 

\subsection{On The Nationalist Spiritual Leaders and Freedom Fighters}
To these two classes, namely "The Nationalist Freedom Fighters" and "The Nationalist Spiritual Leaders", can be assigned, broadly speaking, 
two labels "Violence" and "Non-violence" respectively, not that freedom fighters were not spiritual or that spiritual leaders were not
freedom fighters, but that if we abstract away the details we find that there really were such two classes, I just gave them a name, though
might not be appropriate\footnote{if you think it still is not appropriate to classify this way, feel free to ignore it, it's not
going to affect further reading, I assure you. but make sure you get what my intention was.}.

One name that comes to minds of the most indians, when it's about "Non-violence", is that of Mahatma Mohandas Karamchand Gandhi Bapu. What name would come 
when it's about "violence"? we can't really name one, there were masses doing these acts, but we can name leaders under whom such
acts not only happened but were justified by the leaders themselves on the basis of their philosophies. if we know what caused
gandhi to be non-violent, that of course was Bhagavad Gita, we can ask who was there preaching violence from Bhagavad Gita? then the
name\footnote{I know neither of the two, it's just the history i've read that makes 
me write this, though I understand the history might be wrong! also do note that I do admire certain aspects of Maharaj Tilak, as Gandhi addressed him, 
and I do respect his teachings.} which comes to mind is that of Lokmanya Bal Gangadhar Tilak. for the sake of simplicity let's put Gandhi into the class of 
nationalist spiritual leaders, and Tilak into the class of nationalist freedom fighters. I say that the goal of the latter
class was to gain the political power and to rule the nation on their own principles, while that of former class was to oppose the 
injustice, ask for the human rights, and to bring their own principles into action, on an individual basis.
here I say on an Individual basis, because that's where "Swaraj" of Gandhi differed from Tilak, to Tilak Swaraj 
was political, whereas to Gandhi it was Poetical, He believed in Vedic definition of the word 
"Swaraj", as he defines it,\footnote{
  In \textbf{India of My Dreams}, chapter 2, The Meaning of Swaraj, 
  Hosted at \href{https://www.mkgandhi.org/indiadreams/chap02.htm}{https://www.mkgandhi.org/indiadreams/chap02.htm}
}
(I highly recommend reading all of his definitions of "Swaraj")
\begin{quote}
    The word Swaraj is a sacred word, a Vedic word, meaning self-rule and self-restraint, and not freedom from all restraint which 'independence' often means.
\end{quote}
Gandhi believed in action, so did Tilak, in fact we find Tilak teaching Bhagavad Gita lessons, and elevating and promoting 
especially the Karmayoga over the other yogas. Tilak gave attention to the Bhagavad Gita when he was imprisonment in 
Mandalay in Burma for his controversial articles celebrating Khudiram’s bomb attack on Kingsford, which later got transformed into
his marathi commentary on Bhagavad Gita titled "Srimad Bhagavad Gita Rahasya". On the other hand, Gandhi never supported Tilak, especially in the
case of violence, he read many books when he was in Yerwada jail, including commentaries of Mahabharat and Bhagavad Gita, and 
also that of Tilak, despite this we find Gandhi having completely different views on Gita. He took a bigger picture of Mahabharat to
sketch and understand the context of Gita, while Tilak was hugely basing on the fact that one should fight against the wrong just like
what and how Arjuna fought after the teachings of Krishna. In fact we even find Tilak asking Gandhi to forget about
non-violence and to immediately win "Swaraj", no matter what way, be it violence. Gandhi after getting out of Yerwada, organized sessions on Gita for about nine months at
Sabarmati Ashram in Ahmedabad, which later got transformed into his gujarati commentary on Bhagavad Gita titled "Anaasaktiyog". 

As will be clear in the following essay on philosophy of Karmayoga, as to what exactly caused Gandhi and Tilak to differ, Gandhi had understood the 
purpose of Karmayoga in the context of Bhaktiyoga and Gnanayoga, which makes him see that the ultimate reality is brahamn, and act accordingly,
he seems to have taken the statement "Keep working without the  fruits" as the guiding principle, as he said, 
\begin{quote}
    “I have in all humility felt that perfect renunciation [of fruits] is impossible without perfect observance of ahimsa in every shape and form”.
\end{quote}
It therefore makes sense why Tilak interpreted differently, because to some extent compared to Karma Yoga he ignored\footnote{I feel
like i'm making a strong statement here, just note that all my statements actually attributes to the class which these two representatives 
belong to, I sincerely apologize if i'm making any error or maligning an identity and the pages of history.} 
the Bhakti Yoga and Gnan Yoga. Whatever was the path they chose, they always claimed, and I believe they claimed it right, 
that they were doing "righteous" actions, basing it on their interpretations of Srimad Bhagavad Gita. 

\subsection[On The Philosophy of Karma-Yoga as I understand it]{On The Philosophy of Karma-Yoga as I understand 
it\footnote{Almost all of my understanding comes from the internet, and my childhood education, mostly influenced by 
non-dualistic \href{https://en.wikipedia.org/wiki/Advaita\_Vedanta}{Advaita Philosophy} established by 
\href{https://en.wikipedia.org/wiki/Adi\_Shankara}{Adi Shankaracharya}.}}
Karma-Yoga\footnote{\href{https://en.wikipedia.org/wiki/Karma\_yoga}{https://en.wikipedia.org/wiki/Karma\_yoga}} 
literally means The Yoga\footnote{\href{https://en.wikipedia.org/wiki/Yoga}{https://en.wikipedia.org/wiki/Yoga}} of Karma, 
The Yoga of Action, The Yoga of Work, but what's Yoga? I haven't yet read Patanjali's 
Yogasutras\footnote{\href{https://en.wikipedia.org/wiki/Yoga\_Sutras\_of\_Patanjali}{https://en.wikipedia.org/wiki/Yoga\_Sutras\_of\_Patanjali}},
but as far as i understand it, Yoga means Union, union of what? The self and The 
Brahman\footnote{\href{https://en.wikipedia.org/wiki/Brahman}{https://en.wikipedia.org/wiki/Brahman}}, 
But what is Brahman? Some sort of infinitude of the existence, details of which is beyond the scope of these essays, 
but to get some familiarity with it, here's a verse from Brihadaranyaka 
Upanishad\footnote{\href{https://en.wikipedia.org/wiki/Brihadaranyaka\_Upanishad}{https://en.wikipedia.org/wiki/Brihadaranyaka\_Upanishad}}, 
which mostly discusses Metaphysics of Consciousness,
\begin{quote}
    \begin{sanskrit}
        पूर्णमदः पूर्णमिदं पूर्णात्पूर्णमुदच्यते ।
    \\ पूर्णस्य पूर्णमादाय पूर्णमेवावशिष्यते ॥
    \end{sanskrit}
\end{quote}
Translation\footnote{by Swami Madhavananda, as found \href{https://en.wikipedia.org/wiki/Brihadaranyaka\_Upanishad\#:\~:text=\%22That\%20(Brahman)\%20is,infinite\%20(Brahman)\%20alone.\%22}{here}}
of which in English is,
\begin{quote}
"That (Brahman) is infinite, and this (universe) is infinite. the infinite proceeds from the infinite.
(Then) taking the infinitude of the infinite (universe), it remains as the infinite (Brahman) alone."
\end{quote}
That is, Brahman is the ultimate reality of this existence, though this is not how it's viewed in all
the schools of Hindu Philosophy, at least in the 
Advaita\footnote{\href{https://en.wikipedia.org/wiki/Advaita\_Vedanta}{https://en.wikipedia.org/wiki/Advaita\_Vedanta}} 
(non-dualistic) school of Hindu Philosophy it is the case, and non-dualism is what these essays are based on.

But then the question arises, what does "being union with the brahman" mean and why would one want to be union of this sort? 
it means to be free, to be liberated, to attain 
Moksha\footnote{\href{https://en.wikipedia.org/wiki/Moksha}{https://en.wikipedia.org/wiki/Moksha}}
in Hinduism, and to attain 
Nirvana\footnote{\href{https://en.wikipedia.org/wiki/Nirvana}{https://en.wikipedia.org/wiki/Nirvana}}
in Buddhism, or to attain self-rule (Swaraj) as Gandhi called it. The purpose of all the Yogas, and any religion for that matter, is to be free, 
to have freedom. Freedom? of what kind? and why would one want to have it?
Freedom from Slavery, slavery of the senses, slavery of the identity, or anything that bounds one
to be "selfish", slavery is the cause of selfishness, and selfishness brings into existence a 
property of relation of an action and its effect, called "attachment". so the goal is to have freedom from the attachment 
to effects of an action, or fruits of a karma, but do note that freedom here doesn't mean
relative freedom, that is "freedom from this and then attachment to that", but the absolute freedom,
that is "freedom from this and that", where "this" and "that" refers to any object of this existence,
also note that any subject, which is just one, the self (or aatman), is not an object, and therefore freedom from the subject or
the self is not a thing. this is what Karmayoga is, working without the wish of its results,
doing actions in selflessness, detaching the self from the fruits of karma, identity-less Karma. 
if Karmayoga is to be summarized in one slang, as Swami Vivekananda puts it, "Work for work's sake!".

Why should one renounce the fruits of their own hard work? is it even possible to do an action without
the fruits? what does it even mean to renounce the fruits? if fruits are what makes one slave, shouldn't
"not working at all" solve the problem? why instead to work and then renounce the fruits?
if the freedom of the sort mentioned above is to be attained, there needs to have conscious awareness
of the eternal Brahman, the self, and the Karma one is engaged in.
There are two ways one attains such a freedom, the first is almost impossible, 
that is to renounce the work itself, let alone the fruits of actions because they don't exist in this case.
do not do any actions at all! this feels like a good idea, but is almost impossible, because as Swami Vivekananda 
writes,\footnote{in his poem THE SONG OF THE SANNYÂSIN, hosted \href{https://www.ramakrishnavivekananda.info/vivekananda/volume\_4/writings\_poems/the\_song\_of\_the\_sannyasin.htm}{here}}
\begin{quote}
"Who sows must reap," they say, "and cause must bring
The sure effect; good, good; bad, bad; and none
Escape the law. But whoso wears a form
Must wear the chain." Too true; but far beyond
Both name and form is Âtman, ever free.
Know thou art That, Sannyâsin bold! Say — "Om Tat Sat, Om!"
\end{quote}
any action is a cause
and every cause has an effect, and one can't escape from this effect, at least while existing in this 
existence! as Krishna says in the verses 4 and 5 of chapter 3 of Bhagavad Gita,
\begin{quote}
\begin{sanskrit}
    न कर्मणामनारम्भान्नैष्कर्म्य पुरुषोऽश्न‍ुते ।
\\न च सन्न्यसनादेव सिद्धिं समधिगच्छति ॥ 
\end{sanskrit}

\begin{sanskrit}
    न हि कश्चित्क्षणमपि जातु तिष्ठत्यकर्मकृत् ।
    \\कार्यते ह्यवशः कर्म सर्वः प्रकृतिजैर्गुणैः ॥
\end{sanskrit}
\end{quote}
translation of which is,
\begin{quote}
    Not by merely abstaining from work can one achieve freedom from reaction, nor by renunciation alone can one attain perfection.

    Everyone is forced to act helplessly according to the qualities he has acquired from the modes of material nature; therefore no one can refrain from doing something, not even for a moment.
\end{quote}
It's not, existentially, possible to not to do any action, there are several reasons for this, two of which 
I mention here, first being, what krishna said, "it's the fundamental property of this nature", and
second being "the decision of not doing action is itself an action!". so the work has to be done,
or will be done automatically because that's what the fundamental nature of this existence is, that is
we can not renounce actions, and therefore the only remainning way, the second way, not impossible 
but not esay either, is to renounce the fruits of actions. this is why one should renounce the fruits.

Now comes the beautiful question how to renounce the fruits of karma? the answer, simply, 
is Karma-Yoga! but what is Karma? if one knows what Karma is, then and only then it's actually possible 
to renounce the fruits of Karma. following is the list that defines what karma (work/action) is,
\begin{enumerate}
    \item done by living conscious being
    \item done consciously through identity
    \item done for a particularly purpose
    \item done with moral obligation
    \item done results in effect.
\end{enumerate}
understanding Karma is a bit more complicated than it seems, because we end up with questions of reincarnation,
cause-effect, free-will, and many other BIG Problems in Philosophy. All I can do, at least for the purpose
of these essays, is to take a perspective of Advaita Vedanta. so Karma is something that's an action,
which is a cause and therfore will have its effect, done by a conscious being. what if the 
identity or moral obligation or purpose doesn't exist? this is what Karma-yoga is, it removes these 
internal properties, as Buddha puts, "it removes the second arrow, namely that of internal pain", 
the external properties, the first arrow, however, can't be removed since that's what the nature fundamentally is. 
there are three levels to be free from this Karma-Fruit or Action-Effect, first being renunciation 
of effect, that is work for work's sake, second being renunciation of action itself, which is impossible
as shown before but there's a way theist can do it, by having no purpose or moral obligation,
that is by considering work as worhip. but not all are theist, and so there's a third one, that is to
identify the true self being the Brahman itself!

Karma-Yoga, or any Yoga for that matter, is not possible without renouncing the fruits. and Mahatma 
Gandhi had understood this very well, result of which is his book "Anaasaktiyoga" (literally means 
the science of non-attachment) and his very life he lived.


\subsection{On The BG:3.35 specifically}
Bhagavad Gita is structured into Yogas, one of the important ones, at least the one these essays are concerned with, 
is "Karma Yoga". The sloka BG:3.35, which is as follows,
\begin{sanskrit}
\begin{quote}
श्रेयान्स्वधर्मो विगुणः परधर्मात्स्वनुष्ठितात् ।
\\स्वधर्मे निधनं श्रेयः परधर्मो भयावहः ।।
\end{quote}
\end{sanskrit}
is under the chapter of Karma Yoga, which translates\footnote{from Bhagavad-gītā As It Is by His Divine Grace A.C. Bhaktivedanta 
Swami Prabhupāda, Founder-Ācārya of the International Society for Krishna Consciousness. hosted at 
\href{https://vedabase.io/en/library/bg/}{https://vedabase.io/en/library/bg/}} as,
\begin{quote}
It is far better to discharge one’s prescribed duties, even though faultily, than another’s duties perfectly. 
Destruction in the course of performing one’s own duty is better than engaging in another’s duties, for to follow another’s 
path is dangerous.
\end{quote}
That is, performing own duty is superior to anything. Duty (or own Duty) is defined as an action 
that is both Good and Natural (see table \ref{table:naturalDuty}).

What is Natural Action (Prakruti/Swabhaav Karma)? Natural action is action that is both of Nature and 
Nurture of the self. Whole idea of Karma works only if there's reincarnation, and therefore the 
ancient karma of the self (that is past life karma) is what the nature of the self is in present birth,
every birth carries this natural tendencies from its ancient karmas, but then there's tendencies 
developed through the surrounding in this birth, which is what nurture is, that is Natural action/karma
is what is done because of Prakruti/Swabhaav (Nature and Nurture, or Nature for short) of the self, and even an enlightened is not 
free form it. this can be thought in terms 
of memories, which is how i've understood it, but, it's a bit complicated to dicuss memories and 
its manifestations in these essays given that the scope thereof is just to conclude what Duty is.
another reason I'm not writing about memories here is that it's a bit more abstract concept, and that
I do not yet fully understand it because somehow these memories are not physical like that of 
Computers or Human Brain! as the sloka BG:3.34
states,
\begin{quote}
    \begin{sanskrit}
    इन्द्रियस्येन्द्रियस्यार्थे रागद्वेषौ व्यवस्थितौ ।
    \\तयोर्न वशमागच्छेत्तौ ह्यस्य परिपन्थिनौ ॥ 
    \end{sanskrit}
\end{quote}
translation of which is,
\begin{quote}
    There are principles to regulate attachment and aversion pertaining to the senses and their objects. One should not come under the control of such attachment and aversion, because they are stumbling blocks on the path of self-realization.
\end{quote}
Prakruti gets manifested, first in thoughts, then in speech, and then in action. it's one's own Duty to 
get in control of those and not let them control the self. as is said in BG:3.42,
\begin{quote}
    \begin{sanskrit}
    इन्द्रियाणि पराण्याहुरिन्द्रियेभ्यः परं मनः ।
    \\मनसस्तु परा बुद्धिर्यो बुद्धेः परतस्तु सः ॥
    \end{sanskrit}
\end{quote}
translation of which is,
\begin{quote}
    The working senses are superior to dull matter; mind is higher than the senses; intelligence is still higher than the mind; and he [the soul] is even higher than the intelligence.
\end{quote}
and as depicted in table \ref{table:threeyogas}, the process of getting the control or becoming free or
attaining Swaraj as Gandhi called it, begins with first senses, then the mind, and then the intellect, 
after which the true self is known. 

What is Good Karma? and what is Evil? Good and Evil obviously are subjective, which is exactly why
the goal is not to become happy in life, or to become rich, or any form of good-state-of-emotion, 
neither is it any form of evil-state-of-emotion. happiness or sadness both are a state of slavery,
and Yogis seek freedom from such a slavery. but to detach from fruits while in misery is almost alway 
"feels-like" more misery, which is what Gnan-Yoga, generally speaking, would ask one to do. in Karma-Yoga,
however, there's a slow process, a process through steps to reach that state of detachment. the idea is that 
it's easier to first go through experiences and then renounce, or first become "good" and then 
renounce "being" at any state at all. how does one become "good"? as a verse of Bṛhadāraṇyaka Upaniṣad states, 
\begin{quote}
"Truly, one becomes good through good deeds, and evil through evil deeds". - 3.2.13
\end{quote}
the process of first doing good and then renouncing its fruits is best understood by an example Swami
Vivekananda gives, that is, "There is a thorn in my finger, and I use another to take the first one 
out; and when I have taken it out, I throw both of them aside; I have no necessity for keeping the 
second thorn, because both are thorns after all", the thorn in the finger represents "evil karma",
through "good karma" it is eradicated but a Yogi will not stop there, and further renounce "good karma"
too! that is actually becoming identity-less, or identifying the self as the brahman. but again,
ther's no such thing as good or evil karma, as vivekananda puts it,
\begin{quote}
    He who in good action sees that there is something evil in it, and in the midst of evil sees that there is something good in it somewhere, has known the secret of work.
\end{quote}
so every action is equally good and evil, in a sense as Swami Vivekananda puts it, the sum total of 
good and bad karma of the whole universe (he defines universe as that where, along with other properties
cause-effect is true) is constant. the process of (locally) "doing good" is just a way to reach a
state where renunciation of fruits or identity becomes easier, the harder way, of course, exists and is that 
of Gnaan-Yoga.

What is Duty? Duty, as mentioned earlier, is karma that is both Good and Natural.
there's a way to know own's Duty, it's through asking five questions about following,
\begin{enumerate}
    \item Duty of Station (Ashrama Dharma)
    \item Duty of Profession/Family/Tradition (Varna Dharma)
    \item Conjunction of the above two (Varnaashrama Dharma)
    \item Special Duty (Vishesa Dharma)
    \item General Duty (Sadharana Dharma)
\end{enumerate}
but since above questions are not relevent to modern world, especially because the structure of
Varna-Ashrama no longer exist in india, let alone the world. so then there's a simpler way to find 
one's own Duty, by asking if an action unites the self with the Brahman. or as Swami Vivekananda says,
"Anything that takes one towards GOD realization is one's own Duty".

Mahatma Gandhi had understood Karma-Yoga to that level that he saw non-violence as a necessary condition 
of renunciation, and therefore of any Yogas. by preaching non-violence he was essentially preaching
to renounciate the fruits, because he saw the Duty of his or anyone following him was never that of 
a warrior, which for Arjuna was, but that of a house-holder and working class. this is how, as far as
I understand, Gandhi wanted to establish "Swaraj", the following is one of his definition thereof.
\begin{quote}    
Self-government depends entirely upon our internal strength, upon our ability to fight against the heaviest odds. Indeed, self-government which does not require that continuous striving to attain it and to sustain it is not worth the name. I have, therefore, endeavoured to show both in the word and deed, that, political self-government, that is, Self-government for a large number of men and women, is no better than individual self-government, and therefore, it is to be attained by precisely the same means that are required for individual self-government or self-rule.
\end{quote}

% \newpage

\section{Gujarati Composition of all the four Essays}
Instead of translating, I've combined all the above four essays into one, 
and formed a new coherent short essay that is more simpler and conclusive.

\subsection{\Guja{શ્રીમદ્ભગવદ્ગીતાનો કર્મયોગ અને ભારતીય સ્વતંત્રતા સંગ્રામના આધ્યાત્મિક ક્રાંતિકારીઓ}}

\begin{Guja}
ભારતીય સ્વતંત્રતા ચળવળ એ ભારતમાં બ્રિટીશ શાસનનો અંત લાવવાના અંતિમ ઉદ્દેશ સાથે ચલાવવામાં આવેલી ઐતિહાસિક લડતની ઘટનાઓની શ્રેણી હતી, જેમાં ઘણાં ભારતીયોએ તેમનો ભાગ ભજવ્યો હતો, 
દેશનાં ખૂણે-ખૂણે દેશભક્તિનાં અનેક રૂપો જોવા મળ્યા. આ દેશપ્રેમ સ્વરાજ, સ્વશાસન, સ્વરાષ્ટ્ર, સ્વતંત્ર, અને આઝાદ જેવા ભારતનાં વિશેષણોમાં રૂપાંતર થતો જોવા મળ્યો, અને આવા વિશેષણોને પ્રખ્યાત 
બનાવતા અનેક ક્રાંતિકારી આંદોલનો જોવા મળ્યા, જેના સહભાગીદારીઓનું નેતૃત્વ એ તેમની સફળતાનું કારણ હતું, આવું નેતૃત્વ કરનાર ભારતીય સમુદાયનાં લોકો દેશનાં વિવિધ વર્ગોમાંથી આવતા દેખાયા, ભગત સિંહ 
જેવા યુવાઓનો વર્ગ, સરોજિની નાયડુ જેવા નારીઓનો વર્ગ,  રવીન્દ્રનાથ ટાગોર અને બંકિમચંદ્ર ચટ્ટોપાધ્યાય જેવા લેખકો અને કવિઓના વર્ગથી લઈને અરવિંદ ઘોષ જેવા તત્વચિંતકોનો વર્ગ પણ જોવા મળ્યો. નેતૃત્વ 
કરનાર આ લોકોના જીવનમાં જો થોડું ઊંડું ઉતરીને જોવામાં આવે કે તેમના અંતર્મનમાં શું હતું? જેનાથી તેઓ એ કરી શક્યા જે આપણને અસંભવ જેવું લાગે, તો આપણને દેખાય કે દરેક નેતૃત્વ કરનાર પાસે કોઈ ને કોઈ 
પ્રકારનું તત્વજ્ઞાન જરૂર હતું, આ તત્વજ્ઞાન બીજે ક્યાંયથી નહિ પણ ધર્મગ્રંથોમાંથી આવતું જણાયું, કેટલાક એવા નેતાઓ પણ જોવા મળ્યા જેમના કર્મો ધર્મગ્રંથોથી પરે માનવતાવાદથી થતા, પરંતુ મોટાં ભાગનું ભારતીય 
સ્વતંત્રતા ચળવળનું નેતૃત્વ કર્મોને ન્યાય સમાન ઉચિત ઠરાવવા માટે વિવિધ ધર્મગ્રંથો, ખાસ કરીને શ્રીમદ્ભગવદ્ ગીતાની સહાય લેતું જોવા મળ્યું. શું કરવું જોઈએ અને શું ના કરવું જોઈએ? શું ધાર્મિક છે અને શું અધાર્મિક છે? 
શું સત્ય છે અને શું અસત્ય છે? આવાં પ્રશ્નોના જવાબ જે-તે ધર્મગ્રંથ અને માનવીય જીવનનાં વ્યક્તિગત અર્થઘટનો પર હતું મોટાભાગનું ભારતીય સ્વતંત્રતા ચળવળનું નેતૃત્વ, એકજ ગ્રંથ કે કહેવત બે એવા અર્થઘટનો આપતું 
થયું કે હિંસા અને અહિંસા જેવા અનેક તદ્દન વિરોધી ગુણધર્મો તેમાંથી ઉદ્ભવતા જોવા મળ્યા.

"અહિંસા" સાંભળીનેજ દરેક ભારતીયના મનમાં મહાત્મા મોહનદાસ કરમચંદ ગાંધી બાપુ આવે, પરંતુ "હિંસા" સાંભળીને કોનું નામ આવે મનમાં? આપણે કોઈ એક નામ ન આપી 
શકીએ પરંતુ એવા નેતાઓ કે જેમના હેઠળ ના ફક્ત હિંસા થતી હતી પણ જેમણે ખુદે હિંસાનો પ્રચાર કર્યો હતો એમનું નામ જરૂર લઇ શકાય. એવું એક નામ કે જેમણે ખુદે 
હિંસાને ગીતાના સંદેશા તરિકે બતાવી અને એનો પ્રચાર કર્યો એ લોકમાન્ય બાળગંગાધર ટિળકનું છે. આઝાદીની લડતમાં બે મુખ્ય રીતો જોવા મળી, એક હિંસાની અને બીજી 
અહિંસાની, પહેલી રીતનું લક્ષ્ય "સ્વરાજ", કે જે સ્વયં ટિળકનું જ પ્રચારિત વિશેષણ હતું. પરંતુ આ સ્વરાજ એક રાજકીય સપનું હતું, જ્યારે ગાંધીનું સ્વરાજ 
એક આધ્યાત્મિક સપનું હતું. જેમ ગાંધી સ્વરાજ ની એક 
વ્યાખ્યામાં\footnote{In \textbf{India of My Dreams}, chapter 2, The Meaning of Swaraj, 
Hosted at \href{https://www.mkgandhi.org/indiadreams/chap02.htm}{https://www.mkgandhi.org/indiadreams/chap02.htm}
} 
(હું તેમની "સ્વરાજ" ની દરેક વ્યાખ્યાઓ વાંચવા તમને ભલામણ કરું છું) કહે છે તેમ,
\begin{quote}
સ્વરાજ એક પવિત્ર શબ્દ છે, એ એક વૈદિક શબ્દ છે જેનો અર્થ આત્મ-શાસન અથવા અથવા આત્મ-સંયમ છે, કોઈ બાહ્ય શાસનથી મુક્તિ નહિ કે જે ખરેખર આઝાદીનો અર્થ છે.
\end{quote}
ગાંધી કર્મમાં માનતા હતા, અને ટિળક પણ, ખાસ કરીને ટિળક તો કર્મયોગને બીજા યોગોથી વધારે માન આપતા જોવા મળ્યા. ટિળક જ્યારે કિંગ્સફોર્ડમાં ખુદીરામના બૉમ્બ 
વિસ્ફોટના સમર્થન ભર્યા લેખો લખવાના ગુના બદલ બર્મામાં જેલમાં હતા ત્યારે તેમણે ગીતા તરફ ધ્યાન ધરેલું કે જેણે છેવટે તેમના ભગવદ્ગીતાના મરાઠી ભાષ્ય "શ્રીમદ્ભગવદ્ગીતા 
રહસ્ય" નું રૂપ લીધું. અને જ્યારે ગાંધી યેરવાડા જેલમાં હતા ત્યારે તેમણે ઘણાં પુસ્તકો વાંચ્યા હતા, જુદા-જુદા મહાભારતનાં અને ભગવદ્ગીતાનાં ભાષ્યો વાંચ્યા, જેમાં ટિળકનું 
ભાષ્ય પણ હતું, તેમ છતાં ગાંધીની ગીતાની શીખ ટિળકથી ભિન્ન હતી. જેલમાંથી છૂટયા પછી  ગાંધીએ સાબરમતી આશ્રમમાં નવ મહિના ગીતા પર પ્રવચનો રાખ્યા જેણે છેવટે 
તેમના ભગવદ્ગીતાના ગુજરાતી ભાષ્ય "અનાસક્તિયોગ" નું રૂપ લીધું. ટિળક ફક્ત કર્મયોગને પ્રોત્સાહન આપતા, ગાંધીને હિંસા તરફ વાળતા પણ જોવા મળ્યા, પરંતુ 
ગાંધીએ "કર્મ કરેજા ફળની ઈચ્છા ના રાખ" ને મહત્વનું ઘણી કીધું કે,
\begin{quote}
દરેક રૂપની અહિંસા પાળવા શિવાય મને અનાસક્તિ શક્ય લાગતી નથી.
\end{quote}

શ્રી કૃષ્ણ ભગવાન સાંખ્યયોગ વિશે સમજાવ્યા પછી ગીતાના ત્રીજા અધ્યાયમાં અર્જુનના પ્રશ્ન "જો તમે કર્મ કરતા જ્ઞાનને ચઢિયાતું માનો છો, તો પછી મને આ યુદ્ધરૂપી ભયાનક 
કર્મમાં કેમ જોડો છો?" નો જવાબ કર્મયોગ ના પરિચયથી આપે છે. અર્જુન ઈચ્છે છે કે આ યુદ્ધમાં મારા પોતાનાજ સ્વજનો ને હણીને પાપ નથી કરવો, અને કહે છે કે એવા પાપ થી સુખી ના થવાય, અને ના મને કોઈ 
રાજ્ય ની લાલસા છે. ભગવાન કર્મયોગનો પરિચય નિષ્કર્મતા થી કરે છે, અને કહે છે કે મનુષ્ય કર્મ કર્યા વગર અસ્તિત્વ નથી ધરાવતો. બીજા શબ્દોમાં, ચોથા અધ્યાયના આઢારમાં શ્લોકમાં જેમ કહ્યું છે તેમ "કર્મમાં અકર્મ, 
અને અકર્મમાં કર્મ જેને દેખાય તેજ જ્ઞાની!", આ વિરોધાભાસી વાક્ય ખરેખર અત્યંત મહત્વનું છે, એનો મતલબ થાય છે કે કર્મ કર્યા છતાંય તે કર્મ સ્વયંનું નથી; અર્થાત ફળત્યાગ; એટલે કે અકર્મ અને, કર્મ ન કરવા છતાં; 
અર્થાત કર્મત્યાગ; આવા કર્મત્યાગના નિર્ણયની સ્વતંત્રતા હોવી એતલે કે કર્મ! કર્મતો હંમેશા રહેવાનુંજ.

શ્રીમદ્ભગવદ્ગીતા વિવિધ યોગોથી સંરચિત છે, જેમાંનો એક મહત્વનો, ખાસ કરીને આ નિબંધને જે સંબંધિત છે તે, "કર્મયોગ" ના નામ થી ઓળખાય છે. ત્રીજા અધ્યાયનો શ્લોક પાંત્રીસ, 
\begin{sanskrit}
    \begin{quote}
    श्रेयान्स्वधर्मो विगुणः परधर्मात्स्वनुष्ठितात् ।
    \\स्वधर्मे निधनं श्रेयः परधर्मो भयावहः ।।
    \end{quote}
\end{sanskrit}
જે કર્મયોગનો ભાગ છે અને જેનું ગુજરાતી ભાષાંતર\footnote{from a Personal Copy of \Guja{શ્રીમદ્ભગવદ્ગીતા, ગીતાપ્રેસ, ગોરખપુર}} નીચે પ્રમાણે છે.
\begin{quote}
    સારી રીતે આચરવામાં આવેલા અન્યના ધર્મ કરતાં ગુણ વિનાનો હોવા છતાં પણ પોતાનો ધર્મ ઘણો ચઢિયાતો છે; પોતાના ધર્મમાં તો મરવું પણ કલ્યાણકારી છે, જ્યારે અન્યનો ધર્મ તો ભયપ્રદ છે.
\end{quote}
કર્મને સમજાવતા કેશવ આ શ્લોકમાં કહે છે કે કેવી રીતે પ્રકૃતિ અને ધર્મ એ સ્વધર્મની વ્યાખ્યા આપે છે. પ્રકૃતિ હેઠળ કરેલ, ધાર્મિક હોય કે અધાર્મિક, કર્મને સ્વભાવ કહે છે. 
(જુઓ ટેબલ \ref{table:naturalDuty}) દરેક મનુષ્યને સ્વભાવ હેઠળ કર્મ તો કરવું જ પડે છે, જ્ઞાની હોય કે ના હોય. આમ, જ્ઞાની માટે સ્વભાવ શિવાય કશુંજ કરવાનું થતું નથી, 
જ્યારે અજ્ઞાની તો સ્વભાવને વિપરીત પણ કર્મ કરે છે, એટલે કે અજ્ઞાની મોહ અને પરધર્મ કરતો રહે છે, જ્યારે જ્ઞાની તો ફક્ત પ્રકૃતિમાં રહીને કર્મ કરે છે, પરંતુ હંમેશા પ્રયત્ન કરે છે કે કામમુક્ત પ્રકૃતિકર્મ, 
એટલેકે સ્વધર્મ કરે. માધવ આગળ સમજાવે છે કે સ્વધર્મ, જો અગુણી હોય તેમ છતાં, પરધર્મ કરતા પણ ચઢિયાતો છે, કારણકે પરધર્મ, ગુણી હોય કે ના હોય, એ હંમેશા 
પરપ્રકૃતિ હોય છે. આમ, ભગવાન પ્રકૃતિ મુજબ ધર્મ નિભાવવાનું કહે છે.

"પરંતુ મનુષ્ય પોતે ન ઈચ્છતો હોવા છતાં પણ પરાણે જોડ્યો હોય એમ બીજાથી પ્રેરાઈને પાપનું આચરણ કેમ કરે છે?", કૌન્તેય પૂછે છે. શ્રીજી કામ અને મોહ (જુઓ ટેબલ \ref{table:naturalDuty}) વિષે 
કહે છે કે, ઇન્દ્રિયો, મન, અને બુદ્ધિ એ સર્વ કામનું રહેઠાણ છે, આ કામ આ ઇન્દ્રિયો, મન, અને બુદ્ધિ દ્વારા જ જ્ઞાનને ઢાંકી દઈને જીવાત્માને મોહિત બનાવે છે. સ્થૂળ શરીર કરતાં ઇન્દ્રિયો પરે હોય છે, 
આ ઇન્દ્રિયો કરતા મન પરે હોય છે, મન કરતાં બુદ્ધિ, અને બુદ્ધિ કરતાં પણ પરે જે છે એ આત્મા છે. આમ, આત્માને જાણી બુદ્ધિથી મનને વશ કરી ઇન્દ્રિયોને જીતીને ભગવાન પ્રકૃતિ મુજબ ધર્મ નિભાવવાનું કહે છે.

જેટલું વધુ કર્મયોગને સમજવાનો પ્રયત્ને મેં કર્યો છે એટલો વધુ હું મૂંઝવાતો જાઉં છું, પરંતુ જેટલું કર્મયોગ ને મેં સમજ્યું છે એમ, ખાસ કરીને આદિ શંકરાચાર્યજી નું ગીતાનું ભાષ્ય કહે છે કે, કર્મ અને કર્મયોગ નો ફરક 
પ્રવૃત્તિ અને નિવૃત્તિથી સમજી શકાય, જ્યાં કર્મને ફક્ત પ્રવૃત્તિનો, અને કર્મયોગને નિવૃત્તિ નો એક ભાગ કહ્યો છે. જ્ઞાનયોગ, ભક્તિયોગ, અને કર્મયોગમાં ફક્ત કર્મયોગ એક વ્યક્તિગત નથી, જ્યાં ભક્તિયોગ અને ખાસ 
કરીને જ્ઞાનયોગ વ્યક્તિગત છે. પરંતુ શું છે કર્મયોગ ખરેખર? 
% ત્રણેય યોગને સમજ્યા પછી કર્મયોગનું ખરેખર મહત્વ સમજાય કારણકે, જેમ શંકરાચાર્ય કહે છે તેમ, કર્મયોગ એ જ્ઞાનની શરૂઆત છે. 
કર્મયોગ એટલે કર્મનો યોગ. યોગ એટલે આત્મા અને બ્રહ્મનો યોગ. બ્રહ્મ એ સંપૂર્ણ અસ્તિત્વ છે, બૃહદારણ્યક ઉપનિષદનો આ શ્લોક બ્રહ્મને વ્યાખ્યાયિત કરે છે,
\begin{quote}
    \begin{sanskrit}
        पूर्णमदः पूर्णमिदं पूर्णात्पूर्णमुदच्यते ।
    \\ पूर्णस्य पूर्णमादाय पूर्णमेवावशिष्यते ॥
    \end{sanskrit}
\end{quote}
\begin{quote}
    \begin{Guja}
        અનંત હંમેશાથી અસ્તિત્વ ધરાવે છે અને આ અનંતમાંથી અનંતને બાદ કરી શકાય છે, અને તેમ છતાં અનંત હંમેશા અનંત જ રહેશે, કોઈ ફરક પડશે નહિ, આ અનંતને બ્રહ્મ કહે છે.\footnote{this is my own translation}
    \end{Guja}
\end{quote}
આવા યોગથી કર્મના ફળનો ત્યાગ શક્ય બને છે, જેને ગાંધીએ અનાશક્તિયોગ આપ્યું, અનાસક્તિયોગ ખરેખર તેના મૂળ રૂપમાં "ત્યાગ" છે. ગીતાનો સાર ત્યાગ છે. 
અને આ ગાંધી ગૂઢ રીતે સમજી ગયા હતા. ગાંધીની સ્વરાજની અનેક વ્યાખ્યાઓમાંની એક વ્યાખ્યા આ અનાસક્તિ અર્થાત ફળત્યાગ ને અહિંસા જેવા આંતરિક ગુણધર્મોથી 
દર્શાવતા કહે છે,
\begin{quote}
    \begin{Guja}
        છેવટે સ્વરાજ આપણી આંતરિક શક્તિ પર નિર્ભર કરે છે, મોટામાં મોટી કઠિનાઈઓથી લડવાની આપણી તાકાત. એવું સ્વરાજ જેને પામવા માટે ઘણાં પ્રયત્નો અને સતત જાગૃતિની જરૂર ના હોય, એવા સ્વરાજને સ્વરાજ કહેવું યોગ્ય નથી. મેં વચન અને કાર્યથી એ બતાવવાની કોશિશ કરી છે કે સ્ત્રી-પુરુષોના વિશાળ સમૂહનું રાજનીતિક સ્વરાજ એ દરેકના વ્યક્તિગત સ્વરાજથી વધારે સારું નથી અને એટલેજ તેને પામવાના રસ્તાઓ એ જ છે જે એક વ્યક્તિગત આત્મ-સ્વરાજ અર્થાત આત્મ-સંયમ માટે છે.
    \end{Guja}
\end{quote}
આમ ભારતીય સ્વતંત્રતાનો સંગ્રામ, સ્વધર્મ, અહિંસા, અને સ્વરાજ જેવા વિશેષણોથી ભરપૂર હતો, કે જે ખરેખર કર્મયોગના ખ્યાલો છે.

\end{Guja}

\newpage

\begin{table}
    \centering
    \begin{tabular}{|l|l|l|} 
        \cline{2-3}
        \multicolumn{1}{l|}{} & Good & Evil  \\ 
        \hline
        Nature & Duty & Desire   \\ 
        \hline
        Other's Nature & Other's Duty & Infatuation    \\
        \hline
    \end{tabular}
    \quad
    \begin{tabular}{|l|l|l|} 
        \cline{2-3}
        \multicolumn{1}{l|}{} & \Guja{ધર્મ} & \Guja{અધર્મ}  \\ 
        \hline
        \Guja{પ્રકૃતિ/સ્વભાવ} & \Guja{સ્વધર્મ} & \Guja{કામ}   \\ 
        \hline
        \Guja{પરપ્રકૃતિ} & \Guja{પરધર્મ} & \Guja{મોહ}    \\
        \hline
    \end{tabular}
    \caption{Nature and Morality}
    \label{table:naturalDuty}
\end{table}
 

\begin{table}
    \centering
    \begin{tabular}{cc}
        Gnaanyoga (Purification of Intellect) : & Ignorace $\rightarrow$ Wisdom/Absolute-Freedom \\
        Bhaktiyoga (Purification of Mind) : & Distraction $\rightarrow$ Concentration \\
        Karmayoga (Purification of Senses) : & Slavery $\rightarrow$ Relative-Freedom
    \end{tabular}
    \quad
    \begin{tabular}{cc}
        \Guja{જ્ઞાનયોગ (બુદ્ધિશુદ્ધિ)} : & \Guja{અજ્ઞાન} $\rightarrow$ \Guja{જ્ઞાન/સ્વરાજ}  \\
        \Guja{ભક્તિયોગ (મનશુદ્ધિ)} :    & \Guja{ચંચળતા} $\rightarrow$ \Guja{સ્થિરતા} \\
        \Guja{કર્મયોગ (ઇન્દ્રિયશુદ્ધિ)} : & \Guja{કામ/મોહ} $\rightarrow$ \Guja{સંતુષ્ટતા}
    \end{tabular}
    \caption{Three Yogas of Bhagavad Gita}
    \label{table:threeyogas}
\end{table}

\begin{thebibliography}{99}

    \bibitem{shankara-gita} {\textbf{Adi Shankaracharya, Bhagavadgita Bhashya}, 
\\Translated into English, by Swami Gambhirananda, Advaita Ashrama, Kolkata, 1984, Hosted at \href{https://www.shankaracharya.org/gita\_bhashya.php}
{https://www.shankaracharya.org/gita\_bhashya.php}}

\bibitem{swami-vevekananda} {\textbf{Swami Vivekananda, Complete Works of Swami Vivekananda, Vol 1, Chapter 2, Karma-Yoga}, 
\\Hosted at \href{https://www.ramakrishnavivekananda.info/vivekananda/complete\_works.htm}
{https://www.ramakrishnavivekananda.info/vivekananda/complete\_works.htm}}

\bibitem{asitis} {\textbf{A.C. Bhaktivedanta Swami Prabhupāda, Bhagavad-gītā As It Is}, 
\\Hosted at \href{https://vedabase.io/en/library/bg/}{https://vedabase.io/en/library/bg/}}

\bibitem{gujgita} {\begin{Guja}\textbf{શ્રીમદ્ભગવદ્ગીતા}, ગીતાપ્રેસ, ગોરખપુર.\end{Guja} \\
(Srimad Bhagavad Gita, Gitapress, Gorakhpur)}

\bibitem{gandhi-gita} {{\begin{Guja}\textbf{મોહનદાસ કરમચંદ ગાંધી, અનાસક્તિયોગ}, 
\\નવજીવન પ્રકાશન મંદિર, અમદાવાદ\end{Guja}} 
\\(Mohandas Karamchand Gandhi, Anaasaktiyoga, Navjeevan Prakaashan Mandir, Ahmedabad)}

\bibitem{tilak-rahasya} {\textbf{Bal Gangadhar Tilak, Srimad Bhagavad Gita Rahasya vol 1 \& 2}, 
\\Translated by Bhalchandra Sitaram Sukthankar}

\bibitem{lala-message} {\textbf{Lala Lajpat Rai, The Message Of The Bhagawad Gita}, 
\\United States: Creative Media Partners, LLC, 2018}

\bibitem{Bankim-krishna} {\textbf{Bankim Chandra Chattopadhaya, Krishna Charit}}

\bibitem{vinoba-gita} {\begin{Guja}\textbf{વિનોબા ભાવે, ગીતા-પ્રવચનો}\end{Guja}, 
\\Hosted at \href{https://www.vinobabhave.org/index.php/geeta-pravachan}{https://www.vinobabhave.org/index.php/geeta-pravachan} 
\\(Vinoba Bhave, Gita-Pravachano)
}

\bibitem{aurobindo-essays} {\textbf{Sri Aurobindo , Essays on the Gita}, 
\\vol 19 The complete works of Sri Arurobindo, Sri Aurobindo Ashram Publication Department}

\bibitem{aurobindo-message} {\textbf{Sri Aurobindo, Sri Bhagavad Gita and Its Message}: 
\\With Text Translation and Sri Aurobindo's Commentary, United States: Lotus Press, 1996}

\bibitem{advaita-ny} {Lectures on Advaita Vedanta and Bhagavad Gita by Swami Sarvapriyananda, Vedanta Society of New York 
\\hosted at \href{https://www.vedantany.org}{https://www.vedantany.org}}

\end{thebibliography}

\end{document}
